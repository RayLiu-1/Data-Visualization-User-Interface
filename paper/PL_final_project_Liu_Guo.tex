%-----------------------------------------------------------------------------
%
%               Template for sigplanconf LaTeX Class
%
% Name:         sigplanconf-template.tex
%
% Purpose:      A template for sigplanconf.cls, which is a LaTeX 2e class
%               file for SIGPLAN conference proceedings.
%
% Guide:        Refer to "Author's Guide to the ACM SIGPLAN Class,"
%               sigplanconf-guide.pdf
%
% Author:       Paul C. Anagnostopoulos
%               Windfall Software
%               978 371-2316
%               paul@windfall.com
%
% Created:      15 February 2005
%
%-----------------------------------------------------------------------------


\documentclass[]{sigplanconf}

% The following \documentclass options may be useful:

% preprint      Remove this option only once the paper is in final form.
% 10pt          To set in 10-point type instead of 9-point.
% 11pt          To set in 11-point type instead of 9-point.
% numbers       To obtain numeric citation style instead of author/year.

\usepackage{amsmath}
\usepackage{graphicx}

\newcommand{\cL}{{\cal L}}

\begin{document}

\special{papersize=8.5in,11in}
\setlength{\pdfpageheight}{\paperheight}
\setlength{\pdfpagewidth}{\paperwidth}

\conferenceinfo{CONF 'yy}{Month d--d, 20yy, City, ST, Country}
%\copyrightyear{20yy}
%\copyrightdata{978-1-nnnn-nnnn-n/yy/mm}
%\copyrightdoi{nnnnnnn.nnnnnnn}

% Uncomment the publication rights you want to use.
%\publicationrights{transferred}
%\publicationrights{licensed}     % this is the default
%\publicationrights{author-pays}

%\titlebanner{banner above paper title}        % These are ignored unless
\preprintfooter{short description of paper}   % 'preprint' option specified.

\title{Interactive Word Network Visualization based on Functional Reactive Programming}
%\subtitle{Subtitle Text, if any}

\authorinfo{Fenfei Guo}
           {University of Colorado Boulder}
           {fenfei.guo@colorado.edu}
\authorinfo{Rui Liu}
           {University of Colorado Boulder}
           {rui.liu@colorado.edu}

\maketitle

\begin{abstract}
In this project we implement a Graphical User Interface to interactively visualize the word co-occurrence network using Functional Reactive Programming language. We first learn the definition of Functional Reactive Programming and the essential idea of this programming paradigm; then we learn in practice by designing a suitable application and implementing it with Javascript in a functional programming style. The interface supports the functions of visualizing user selected networks and passing user feedbacks to back-end programs. 
\end{abstract}


\keywords
Functional Reactive Programming, Javascript, Graphical User Interface, Network Visualization


\section{Introduction}
\label{sec:intro}





\section{Background Knowledge}
\label{sec:back}




\section{Application Design}
\label{sec:app}

\section{Implemenation Details and Demonstration}
\label{sec:impl}


\subsection{Javascript}

\section{Conclusion}
\label{sec:conc}


% We recommend abbrvnat bibliography style.

\bibliographystyle{abbrvnat}

% The bibliography should be embedded for final submission.

\begin{thebibliography}{}
\softraggedright

\bibitem[Smith et~al.(2009)Smith, Jones]{smith02}
P. Q. Smith, and X. Y. Jones. ...reference text...

\end{thebibliography}


\end{document}
