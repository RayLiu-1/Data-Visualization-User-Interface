%-----------------------------------------------------------------------------
%
%               Template for sigplanconf LaTeX Class
%
% Name:         sigplanconf-template.tex
%
% Purpose:      A template for sigplanconf.cls, which is a LaTeX 2e class
%               file for SIGPLAN conference proceedings.
%
% Guide:        Refer to "Author's Guide to the ACM SIGPLAN Class,"
%               sigplanconf-guide.pdf
%
% Author:       Paul C. Anagnostopoulos
%               Windfall Software
%               978 371-2316
%               paul@windfall.com
%
% Created:      15 February 2005
%
%-----------------------------------------------------------------------------


\documentclass[]{sigplanconf}

% The following \documentclass options may be useful:

% preprint      Remove this option only once the paper is in final form.
% 10pt          To set in 10-point type instead of 9-point.
% 11pt          To set in 11-point type instead of 9-point.
% numbers       To obtain numeric citation style instead of author/year.

\usepackage{amsmath}
\usepackage{graphicx}

\newcommand{\cL}{{\cal L}}

\begin{document}

\special{papersize=8.5in,11in}
\setlength{\pdfpageheight}{\paperheight}
\setlength{\pdfpagewidth}{\paperwidth}

\conferenceinfo{CONF 'yy}{Month d--d, 20yy, City, ST, Country}
%\copyrightyear{20yy}
%\copyrightdata{978-1-nnnn-nnnn-n/yy/mm}
%\copyrightdoi{nnnnnnn.nnnnnnn}

% Uncomment the publication rights you want to use.
%\publicationrights{transferred}
%\publicationrights{licensed}     % this is the default
%\publicationrights{author-pays}

%\titlebanner{banner above paper title}        % These are ignored unless
\preprintfooter{short description of paper}   % 'preprint' option specified.

\title{Interactive Word Network Visualization based on Functional Reactive Programming}
%\subtitle{Subtitle Text, if any}

\authorinfo{Fenfei Guo}
           {University of Colorado Boulder}
           {fenfei.guo@colorado.edu}
\authorinfo{Rui Liu}
           {University of Colorado Boulder}
           {rui.liu-1@colorado.edu}

\maketitle

\begin{abstract}
In this project we implement a Graphical User Interface to interactively visualize the word co-occurrence network using Functional Reactive Programming language. We first learn the definition of Functional Reactive Programming and the essential idea of this programming paradigm; then we learn in practice by designing a suitable application and implementing it with Javascript in a functional programming style. The interface supports the functions of visualizing user selected networks and passing user feedbacks to back-end programs. 
\end{abstract}


\keywords
Functional Reactive Programming, Javascript, Graphical User Interface, Network Visualization


\section{Introduction}
\label{sec:intro}

 In class, we've learned the basic usage of \texttt{ Ocaml} and the functional programming paradigm. This paradigm has lots of advantages and has many deviations while being applied in the real word applications. One of the major deviations that benefits from the functional programming paradigm is the  Functional Reactive Programming (FRP) paradigm that provides an elegant way to express computation in domains such as interactive animations, robotics, computer vision, user interfaces, and simulations, etc.

In  this project we implement a Graphical User Interface to interactively visualize the word co-occurrence network using Javascript and follow the FRP paradigm. The primary goal of this project is to learn what is Functional Reactive Programming, and what is the benefit of this programming paradigm by implementing a simple application.

Graphical User Interface design and implementation is a technique that is so widely used that we benefit from it everyday while using computers. Therefore we decide to implement an user interface using one of the most popular language that supports at the same time object-oriented, imperative, and functional programming styles, i.e., Javascript.

Data science has gained lots attentions in recent years and has achieved success in many real word applications. The data science is not a closed black-box that we just need to through data into it and wait for a golden solution. To develop better techniques for making predictions and to include human in the decision loop, we need to analyze the data itself. Besides of the statistics tools and algorithmic  analysis, it is often very useful to have a visualization of the data so that we can capture intuitively the intrinsic nature of the data.

Therefore, we decide to implement a Graphical User Interface for interactively visualizing data. The specific application is rooted from one of my own research project.  I'm working on word related tasks such as lexical substitution where the model should provide substitutes for a target in a given sentences, such as "I ate an \underline{apple} this morning". We'd like to gain user feedbacks about the model selections. It would be useful if the user can gain more information about the target words and model selections, which could be the co-occurrence network of a certain word in training corpus.

The major technical problems we need to solve in this project includes:

\begin{itemize}
	\item Find a framework and visualize the networks in user interface.
	\item Exchange data between the front-end interface and the back-end programs (substitution model).
	\item Build an interface so that it can properly show information to the user and collect user feedbacks.
\end{itemize}


Our project is thus can be divided into the following steps:

\begin{enumerate}
	\item Learn about FRP and FRP languages.
	\item Application Design.
	\begin{enumerate}
		\item Interface function design
		\item Interface design
		\item Pipeline design
	\end{enumerate}
	\item Implementation
	\begin{enumerate}
		\item Back-end and web sever
		\item Front-end
	\end{enumerate}
\end{enumerate}

In the report, we first explain the definition and properties of Functional Reactive Programming paradigm in Section \ref{sec:back} \textit{Background Knowledge}; then briefly describe the pipeline of our application and the design of each part in Section \ref{sec:app} \textit{Application Design}.  In Section \ref{sec:impl}, we describe the implementation details along with the demonstration of our interface. We then conclude our work and what we learned in Section \ref{sec:conc}.



\section{Background Knowledge}
\label{sec:back}




\section{Application Design}
\label{sec:app}

In this section, we introduces the basic functions of our application and the design of it.


\subsection{Functions}
\label{sec:func}
The purpose of this application is to interactively visualize word co-occurrence networks and collecting feedbacks from user. Therefore, the application should have these basic functions:

\begin{enumerate}
	\item Pass questions and network information from back-end program to the front-end interface.
	\item User interface functions:
	\begin{enumerate}
		\item Display the questions and answers.
		\item Support selection behavior, and return the user selections.
		\item Display the visualization of word networks of user selected words
		\end{enumerate}
		
	\item Pass user feedbacks to the back-end and generate new data based on user feedbacks.
\end{enumerate}


\subsection{User Interface Design}


To include the functions we describe in Section \ref{sec:func}, we divide the major part of the user interface into two parts, the questions area and the visualization area.

\subsubsection{Overall Interface}

Figure \ref{fig:net_demo} shows the overall design of our interface. 

\begin{figure}[h]
\centering
\includegraphics[width=0.95\linewidth]{figure/net_demo}
\caption{Demonstration of the overall user interface for interactively word network visualization}
\label{fig:net_demo}
\end{figure}


\subsubsection{Question Area}

The question area on left in Figure \ref{fig:net_demo} is the major area that user can make interactions, including making selections, picking one word to visualize the network and submitting their feedbacks. 

\begin{figure}[h]
\centering
\includegraphics[width=0.8\linewidth]{figure/qs_demo}
\caption{Question Area}
\label{fig:qs_demo}
\end{figure}


Users and click on the \textit{check-box} element to select the answers, and make a submission by clicking on the submit \textit{button}.

Users can also chose to see the word network of a specific answer by just clicking on the word. They can also switch the graph back to the original target word network by clicking on the question. Figure \ref{fig:net_demo_new} shows the new network graph after user clicking on the word "deserving".

\begin{figure}[h]
\centering
\includegraphics[width=\linewidth]{figure/net_demo_new}
\caption{Visualization of a new word network selected by user}
\label{fig:net_demo_new}
\end{figure}


\subsection{Pipeline design}


The pipeline for our project is very simple, it includes three parts, and we describe more implementation details in Section \ref{sec:impl}.

\begin{enumerate}
	\item Back-end: to generate data. We use python to implement it.
	
	\item Web framework: to exchange data between back-end and front-end by post and request, and to deploy it online and control the routing. We use flask to implement it.
	
	\item Front-end: the user interface. We use Javascript + HTML + CSS to implement it.
\end{enumerate}


\section{Implemenation Details and Demonstration}
\label{sec:impl}
We implement both a backend program, a server program and a frontend program for our project.  The backend consists of a data training program and data transferring program. We build our data training program by Python, which is composed of two parts. The first part is a data training algorithm that provides data that used for network generation. This data training algorithm will output a certain sentence with a underlined word, several alternative words and network data that encapsulated as JSON data.  The network data include server networks the frequency that the word co-occurs with other words in the training corpus. The server is made of Flask, which is a microframework for Python. The server takes the output of backend program as input, and transfer the data to front end when the front end asks. At last but not least, the front end is the part with the function of user interface and data visualization. It asks the server to get the data generated by the backend.  We develop the main function of front end by Javascript. The relationship of the three parts is shown in Figure 5. 

\begin{figure}[h]
\centering
		\includegraphics[width=\linewidth]{figure/workflow}
	\caption{workflow of Interactive Word Network Visualization}
	\label{fig:frp_demo}
\end{figure}

As is shown in Figure 5, the browser request the data of sentence, words, and network when the web page is loaded; then the server pass this request to the backend; the backend response with related data and the server pass it to the front end. 
The front end will request data with special parameter when the user chose some words and click submit button. Then the server pass the request to the backend. The backend will push some data new that related with the parameter. The server aslo pass this data. At last, the front end will show new networks related with the user's last choice.
\subsection{Javascript}

We use HTML, CSS and JavaScript build our frontend program. HTML and CSS describe the structure of the web page. We implement the user interface and data visualization function by JavaScript. JavaScript is a high-level, dynamic, untyped, and interpreted programming language.\cite{JS11} It is a multi-paradigm language, supports functional programming styles. We use JavaScript to implement this project because it is used such that we believe it is meaningful for computer engineer and computer scientist to learn it. What's more,  Java is an open source language such that all the best engines, tools, and libraries are open source. We can find useful functional program libraries. Although it is not a purely functional programming language, we can develop a functional styles program with functional programming libraries. In this project, we development our User Interface part by Lazy.js and the data visualizatoin part by Vis.js.

\subsubsection{Lazy.js}
Lazy.js is a utility library in a similar vein to Underscore or Lo-Dash, but with lazy evaluation.  Underscore or Lo-Dash provide a host of useful functions for array transformation: each function accepts an array as input,  does something with it and gives back a new array. However, the core of Lazy.js is function composition: each function accepts a function as input, stores it, and gives back an object that can do the same. 

It effectively changed the behavior of the input function, and that is why the performance of Lazy.js is better than Underscore and Lo-Dash. 
\begin{figure}[h]
\centering
		\includegraphics[width=\linewidth]{figure/lazy-performance}
	\caption{Lazy.js performance versus Underscore and Lo-Dash}
	\label{fig:lazy-performance}
\end{figure}

In this project, we mainly use two Lazy.js 's function: "map" and "reduce" to build some functional programming code. The map can create a new sequence whose values are calculated by passing this sequence's elements through some mapping function.  
Reduce can aggregate a sequence into a single value according to some accumulator function. 
\begin{figure}[h]
\centering
		\includegraphics[width=\linewidth]{figure/mapreduce}
	\caption{code of Lazy.map and Lazy.reduce}
	\label{fig:mapreduce}
\end{figure}

As is shown in Figure 6, this code generates some checkbox and choice words,  where clicking the words will trigger a JavaScript function that will change the network. 
The function "setanswer()" takes a string(word) as input and return some HTML code that generates a checkbox and shows the words on the web page. The function "strcat()" catches two string. Thus "Lazy(answer).map(setanswer)" outputs the checkboxes and words the all the words in array "answer". Then we apply it with "Lazy().reduce(strcat)" to catch the HTML code for all words. The last step is embedding the code generate by "reduce" into HTML element with id "answer". The whole process do not influence other JavaScript variable. In other words, it is some functional programming code.

\subsubsection{Vis.js}
Vis.js is a dynamic, browser-based visualization library, which is easy to use and enable handling large amounts of dynamic data and manipulation the interaction with data.


\begin{figure}[h]
\centering
		\includegraphics[width=\linewidth]{figure/VisNet}
	\caption{Network generated by Vis.js}
	\label{fig:netvis}
\end{figure}

We call "vis.Network()" funtion in our program when we want to show the network(such as the network in Figure 8) or refresh it. 


\begin{figure}[h]
\centering
		\includegraphics[width=\linewidth]{figure/Vis}
	\caption{code of Vis.network}
	\label{fig:vis}
\end{figure}

"vis.Network" is a visualization to display networks consisting of nodes and edges. As is shown in Figure 9, the function "vis.Network" takes 3 input parameters. The first one is container which is a CSS frame container that describes the position of the network. The second is the network data that parsed from JSON data imported from the server. The last parameter is the network option. Since our network is simple, we only need to set the shape of nodes as 'dot'. The front end program will generate a network on the CSS container with Id "mynetwork".

\subsection{Flask}

Flask is a microframework for Python. The "micro" in microframe work means Flask aims to keep the core simple but extensible.

\begin{figure}[h]
\centering
		\includegraphics[width=\linewidth]{figure/FlaskHttp}
	\caption{Flask HTTP Method}
	\label{fig:flshttp}
\end{figure}
As is shown in Figure 10, the server will run this part of code when it receive a HTTP GET request with route Host/data/$\langle$para$\rangle$, where $\langle$para$\rangle$ is used as parameter that can be passed by the function "get$\_$data()", which is a function work with backend and return the data gotten from backend.

\begin{figure}[h]
\centering
		\includegraphics[width=\linewidth]{figure/FlaskRun}
	\caption{Flask run server}
	\label{fig:flsrun}
\end{figure}

The Figure 11 shows how flask run a server on a certain port. As is show, the function "app.run(port)" runs the program as a server on the port.

\section{Conclusion}
\label{sec:conc}
In this project,we build back-end and web server by python. We implemented data training algorithem in back-end program and build the server with flask. What's more, the front-end is build in functional reactive programming paradigm. We learned something that is very new to us. Due to the time limitation, our programs is simple and with limit functions. However, we believe that we learned a lot of new technologies that is very useful.
\subsection{User Interface}

This project is the first time to develop an User Interface program for both of us. It is a valuable experience for us. Comparing with other types of projects, User Interface not only need implement correct logic funtion, but also think about how to make the interface beautiful and humane. Due to the time limitation, we learned about how to construct web page by CSS and HTML. However, we found that the time is not enough for us when we feel we can handle CSS and HTML.

\subsection{Reactive Programming}

We parsed the data stream that thenetwork information, the questions and answers, and the feedbacks from the user. We use JavaScript to implement reactive paradigm. First, the User Interface JavaScript pragram parsed the input data into an JSON object. Second, the feedbacks from the user are encapsulated into a string and are passed to server.

\subsection{Functional Programming}

We use Lazy.js to program JavaScript in functional programming style. This is also the first time that we develop a project by functional programming. We have a deeper understanding on functional programming after this project. For example, we have discuss the recursion expression in Language T. The reduce method of Lazy.js is similar with recursion. What's more, we learned that functional programming  languages use side effect free functions as a basic building block. This makes the program thread safety, easy to re-purpose and enable concurrency. Program that functionally programmed is easier to trace, debug and test. However, the lack of side effects make IO hard to implement. Thus, we also use some nonfunctional feature in our project. In this project, we use functional programming code to process the data and use nonfunctional programming code to parsed the input data and show the computing results in HTML.
\subsubsection{Lazy Evaluation}

Lazy evaluation is an evaluation strategy which delays the evaluation of an expression until its value is needed and also avoids repeated evaluations.\cite{pldc10} Lazy.js support reactive programming by Lazy evaluaton. What's more Lazy evaluation make the program lighter: the program will run without pre-process that it improve the performanve by avoiding needless calculations, and error conditions in evaluating compound expressions. This project help us understand Lazy evaluation deeply. We believe that it will be used in our future data process study.


% We recommend abbrvnat bibliography style.

\bibliographystyle{abbrvnat}

% The bibliography should be embedded for final submission.

\begin{thebibliography}{}
\softraggedright

\bibitem[Smith et~al.(2009)Smith, Jones]{smith02}
P. Q. Smith, and X. Y. Jones. ...reference text...


\bibitem{JS11} 
Flanagan, David. 
\textit{JavaScript: The Definitive Guide (6th ed.)  }O'Reilly & Associates.
 2011.
 
 
\bibitem[ David Anthony Watt; William Findlay (2004)]{pldc10} 
David Anthony Watt; William Findlay

[\textit{Programming language design concepts}]. 
John Wiley and Sons, pp.367-368,  ISBN 978-0-470-85320-7.

\end{thebibliography}


\end{document}
